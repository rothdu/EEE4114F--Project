% ----------------------------------------------------
% Literature Review
% ----------------------------------------------------
% \documentclass[class=report,11pt,crop=false]{standalone}
% \input{../Style/ChapterStyle.tex}
% \input{../FrontMatter/Glossary.tex}
% \begin{document}
% \ifstandalone
% \tableofcontents
% \fi
% % ----------------------------------------------------
\section{Core Objective \label{ch:Objective}}
% \epigraph{If you wish to make an apple pie from scratch, you must first invent the universe.}%
%     {\emph{---Carl Sagan}}
\vspace{0.5cm}
% ----------------------------------------------------

In sections \ref{ch:Feature_Extraction} and \ref{ch:ML_Methods}, a general hypothesis was outlines regarding the specific questions that should be answered in section \ref{ch:T_R}. 

\begin{itemize}
    \item Does the depth of the neural network affect the model performance? This was analysed by comparing the networks on the three types of spectrograms with different windowing methods.
    \item How significant is the effect of spectral leakage on the performance of the model? This was analysed by comparing the three types of spectrograms on the `shallow' network.
    \item What is the performance of MFCCs on different networks? 
\end{itemize}

Apart from the described changes to the MFCC network, hyperparameter tuning for the models was considered beyond the scope of the investigation. Most models discussed could likely be improved with hyperparameter tuning, and similarly could be improved with a larger dataset.

% ----------------------------------------------------
% \ifstandalone
% \bibliography{../Bibliography/References.bib}
% \printnoidxglossary[type=\acronymtype,nonumberlist]
% \fi
% \end{document}
% ----------------------------------------------------