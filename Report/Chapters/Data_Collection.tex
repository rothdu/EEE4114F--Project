% ----------------------------------------------------
% Data Collection
% ----------------------------------------------------
\documentclass[class=report,11pt,crop=false]{standalone}
\input{../Style/ChapterStyle.tex}
\input{../FrontMatter/Glossary.tex}
\begin{document}
\ifstandalone
\tableofcontents
\fi
% % ----------------------------------------------------
\section{Data Collection \label{ch: data_collection}}

As mentioned previously, four instruments were selected to train the model. Sound clips were collected for the Snare, Trumpet, and Violin, from \href{https://freesound.org/}{freesound} \cite{freesound} and \href{https://pixabay.com/}{pixabay} \cite{pixabay}, which house massive libraries of royalty free music. It is very important to collect a variety of datasets that are unfiltered to resemble real-world applications. The piano sound files were collected by recording short songs played through computer speakers on \href{https://tytel.org/helm/}{Helm} \cite{tytel2018Helm}, an open-source digital synthesiser.

70-100 samples were collected for each instrument, where the data for each instrument is initially split into 80\% training and 20\% test data, with the training data further split into using into training and validation data using varying techniques at different stages.

% ----------------------------------------------------
\ifstandalone
\bibliography{../Bibliography/References.bib}
\printnoidxglossary[type=\acronymtype,nonumberlist]
\fi
\end{document}
% ----------------------------------------------------